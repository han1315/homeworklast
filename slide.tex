\documentclass{beamer}        

\usepackage{ctex}
\usepackage{graphicx}
\usepackage{mathtools}
\usepackage{listings}
\usepackage{amsmath,bm,amsfonts,amssymb,enumerate,epsfig,bbm,calc,color,ifthen,capt-of,multimedia,hyperref}
\usepackage{xeCJK}
\setCJKmainfont{WenQuanYi Micro Hei Mono}

\usetheme{Frankfurt}
\usecolortheme{orchid}
\graphicspath{{image/}}

\begin{document}

\title{项目作业:Julia集的探索和分析}
\author{洪晨瀚 \\ 信息与计算科学 3200300133} 
\institute{数学科学学院}
\date{04-07-2022}
\frame{\titlepage}


\begin{frame}{目录}
  \tableofcontents
\end{frame}

\section{前言}
\begin{frame}{前言}
  Julia集合是一个在复平面上形成分形的点的集合。以法国数学家加斯顿·朱利亚的名字命名。Julia集合通过普普通通的函数生成各种美丽而繁复的图形,往往让人惊叹不已。Julia集合和Mandelbrot集合生成的分形图形各具特点且美轮美奂,十分吸引人。Julia集合是由迭代产生,而迭代是不断重复的过程。在数学上,该过程往往是指计算某个数学函数。Julia集合与Mandelbrot集合一样,被迭代的都是二次多项式$f(x)=x^2+c$。通过迭代该函数而产生的轨迹,具有极大的研究意义,对迭代理论的研究影响深远。 

\end{frame}

\section{数学理论}
\begin{frame}{数学理论}
  \begin{itemize}
  \item Julia集合与Mandelbrot集合都能由$f(x)=x^2+c$反复迭代得到。
  \item Julia集合:固定$c$的值,取$z=z_0$,从而迭代序列为$(0,f_c(0),f_c(f_c(0),f_c(f_c(f_c(0)),\cdots)$。
  \item Mandelbrot集合:$z=0$,$c$为复数参数,从而迭代序列为$(0,f_c(0),f_c(f_c(0),f_c(f_c(f_c(0)),\cdots)$。
  \end{itemize}
\end{frame}  

\section{算法}
\subsection{Julia集合的算法}
\begin{frame}[fragile]{Julia集合的算法}
  \begin{block}{Julia集}
\begin{verbatim}
  for each z in complex ,|c|<2,count=0
  do z=z^2+c,count+=1
  loop until abs(z)>2 || count>maxcount
  if count>maxcount ,draw c,black
  else draw c,white
\end{verbatim}
  \end{block}
\end{frame}

\subsection{Mandelbrot集合的算法}
\begin{frame}[fragile]{Mandelbrot集合的算法}
  \begin{block}{Mandelbrot集}
\begin{verbatim}
  for each c in complex ,z=0,count=0
  do z=z^2+c,count+=1
  loop until abs(z)>2 || count>maxcount
  if count>maxcount ,draw c,black
  else draw c,white
\end{verbatim}
  \end{block}
\end{frame}

\section{数值算例}
\subsection{Julia集合例子}
\begin{frame}{Julia集合例子}
\begin{figure}[H]
  \centering
  \begin{minipage}[t]{0.48\textwidth}
    \centering
    \includegraphics[width=4cm]{a}
    \caption{Julia 1}
  \end{minipage}
  \begin{minipage}[t]{0.48\textwidth}
    \centering
    \includegraphics[width=4cm]{b}
    \caption{Julia 2}
  \end{minipage}
\end{figure}

\begin{figure}[H]
  \centering
  \begin{minipage}[t]{0.48\textwidth}
    \centering
    \includegraphics[width=4cm]{c}
    \caption{Julia 3}
  \end{minipage}
  \begin{minipage}[t]{0.48\textwidth}
    \centering
    \includegraphics[width=4cm]{d}
    \caption{Julia 4}
  \end{minipage}
\end{figure}  
\end{frame}

\subsection{Mandelbrot集合例子}
\begin{frame}{Mandelbrot集合例子}
\begin{figure}[H]
  \centering
  \begin{minipage}[t]{0.48\textwidth}
    \centering
    \includegraphics[width=4cm]{z}
    \caption{Mandelbrot 1}
  \end{minipage}
  \begin{minipage}[t]{0.48\textwidth}
    \centering
    \includegraphics[width=4cm]{y}
    \caption{Mandelbrot 2}
  \end{minipage}
\end{figure}

\begin{figure}[H]
  \centering
  \begin{minipage}[t]{0.48\textwidth}
    \centering
    \includegraphics[width=4cm]{x}
    \caption{Mandelbrot 3}
  \end{minipage}
  \begin{minipage}[t]{0.48\textwidth}
    \centering
    \includegraphics[width=4cm]{w}
    \caption{Mandelbrot 4}
  \end{minipage}
\end{figure}
\end{frame}

\section{结语}
\begin{frame}{结语}
  不同的参数$c$可能使序列的绝对值逐渐发散到无限大,也可能收敛在有限的区域内。Mandelbrot集合是使序列不延伸至无限大的所有复数$c$的集合,而将使其不扩散的$z$值的集合称为Julia集合。Mandelbrot集的神奇之处就在于,我们可以对这个分形图形不断放大,不同的尺度下所看到的景象可能完全不同。放大到一定时候,我们甚至可以看到更小规模的Mandelbrot集,这证明Mandelbrot集是自相似的。而不同的复数$c$对应着不同的Julia集,也就是说每取一个不同的$c$我们都能得到一个不同的Julia集分形图形,并且令人吃惊的是每一个分形图形都是那么美丽。
\end{frame}

\begin{frame}
\begin{center}
\begin{minipage}{1\textwidth}
\setbeamercolor{mybox}{fg=white, bg=black!50!blue}
\begin{beamercolorbox}[wd=0.70\textwidth, rounded=true, shadow=true]{mybox}
  \LARGE \centering 谢谢观看! 
\end{beamercolorbox}
\end{minipage}
\end{center}
\end{frame}

\end{document}
