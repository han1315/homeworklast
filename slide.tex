\documentclass{beamer}        

\usepackage{graphicx}
\usepackage{mathtools}
\usepackage{amsmath,bm,amsfonts,amssymb,enumerate,epsfig,bbm,calc,color,ifthen,capt-of,multimedia,hyperref}
\usepackage{xeCJK} 
\setCJKmainfont{WenQuanYi Micro Hei Mono}

\usetheme{Frankfurt}
\usecolortheme{orchid}


\begin{document}

\title{项目作业:Julia集的探索和分析}
\author{洪晨瀚 \\ 信息与计算科学 3200300133} 
\institute{数学科学学院}
\date{04-07-2022}
\frame{\titlepage}

%% --> 目录结构
%
\begin{frame}{目录}
  \tableofcontents[hideallsubsections]
\end{frame}

%% --> 正式内容开始
%
\section{文学}    % 第 1 节

%% 每一节开头显示目录,并高亮当前节的主题
\AtBeginSection[]{\frame{\tableofcontents[currentsection,hideallsubsections]}}

%% --> 第 1 帧
\begin{frame}{文学}{论语}

\onslide<1->{子在川上曰:逝者如斯夫,不舍昼夜!}

\onslide<2->{子曰:巧言令色,鲜矣仁!}

\end{frame}

%% --> 第 2 帧
\begin{frame}{中外文学}{Shakespere}

To be, or not to be, that is the question.

\end{frame}


\section{数学}    % 第 2 节

%% --> 第 3 帧
\begin{frame}[t]{数学}{欧拉公式}  % 取消垂直居中

欧拉发现了下面的公式
$$
e^{ix} = \cos x + i \sin x
$$
当$x = \pi$时,得
$$
e^{\pi i} + 1 = 0
$$

\end{frame}

\end{document}
